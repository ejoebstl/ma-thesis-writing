%% LaTeX2e class for student theses
%% sections/content.tex
%% 
%% Karlsruhe Institute of Technology
%% Institute for Program Structures and Data Organization
%% Chair for Software Design and Quality (SDQ)
%%
%% Dr.-Ing. Erik Burger
%% burger@kit.edu
%%
%% Version 1.3.2, 2017-08-01

\chapter{ASR with HMM-based systems}
\label{ch:HMM_ASR}

This chapter will be focused on how ASR is done with Janus. It will contain: 
\begin{itemize}
    \item A brief introduction to HMM models
    \item A brief introduction to HMM-based ASR tools:
    \begin{itemize}
        \item description of n-gram language models, dictionaries and context-dependent phone models
        \item description of the purpose of an acoustic model
        \item explanation, about how the this component are combined to form a speech recognizer 
        \item example, showing how the language and phone models, as well as the dictionary are combined to form a HMM
    \end{itemize}
    \item A description of the Word-Error-Rate and Frame-Error-Rate metric. 
    \item An explanation about training HMM-based systems using the expected maximisation algorithm. 
\end{itemize}

\chapter{Time Delay Neural Networks}
\label{ch:TDNN}
The goal of this chaper is to provide a short introduction to neural networks, 
as well as explain the concept of TDNNs. It will contain:
\begin{itemize}
    \item A brief intro to MLPs and SGD
    \item Parameter coupling (convolutional neural networks)
    \item Time delay neural networks
    \item Interpretation of TDNNs as FIR filters
\end{itemize}

\chapter{Acoustic Modelling using Neural Networks}
\label{ch:acoustic_modelling}
The goal of this chapter is to describe the approach of using DNNs for acoustic modelling.
The contents will be: 
\begin{itemize}
    \item definition of the acoustic model training as a deep learning problem
    \item different discriminative training strategies
    \begin{itemize}
        \item Binary Cross Entropy loss on existing labels
        \item Bianry Cross Entropy with re-generating the labels, then trianing again
        \item Minimum Bayes Risk and variants, especially State-Minimum-Bayes-Risk
    \end{itemize}
    \item A brief section about common tricks used when trianing DNN acoustic models, 
          especially Exponential Decay/Newbob.
    \item If there is time left: A brief analysis of the 2nd derivative of the loss function
    during gradient descend.
\end{itemize}

\chapter{Experiment Setup}
\label{ch:experiment_setup}
This chapter should describe:
\begin{itemize}
    \item how the training was structured
    \item which data, and which preprocessing was used
    \item how was the data reverbed
    \item which learning rate schedulers, optimizers, loss functions were used
    \item which mechanisms were used to make the training faster and scalable
\end{itemize}
