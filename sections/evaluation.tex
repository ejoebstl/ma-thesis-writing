%% LaTeX2e class for student theses
%% sections/evaluation.tex
%% 
%% Karlsruhe Institute of Technology
%% Institute for Program Structures and Data Organization
%% Chair for Software Design and Quality (SDQ)
%%
%% Dr.-Ing. Erik Burger
%% burger@kit.edu
%%
%% Version 1.3.2, 2017-08-01

\chapter{Evaluation on Reverberated Data}
\label{ch:results}
This section contains the final evaluation. It describes how different models performed on reverberated data.
\section{Neural Network Models}
We compare the four-layer TDNN model with a fully connected baseline model that was also used in \cite{nguyen20162016}. \\ \\
\subsection{TDNN Model}
The TDNN model, which can be seen in figure \ref{fig:final_tdnn} is based on the results found in Chapter \ref{ch:tdnn_design}. It consists of four convolutional layers and one linear layer at the end, followed by a soft max nonlinearity. It uses the L2-Pooling nonlinearity with zeroing of unstable gradients and a pool size of 10. A splicing layer with the splicing configuration $(0, 1, 2, 3, 3, 4, 5, 6)$ was used after the first TDNN layer. The exact configuration of kernel sizes and strides can be found in table \ref{tbl:tdnn_layer_design}. The output of each TDNN layer had 3000 channels, the output of each pooling layer 300 channels. The input context consisted was $(-13, 9)$. The count of channels and the omission of batch normalization was the main difference to the architecture described in \cite{peddinti2015reverberation} and \cite{peddinti2015jhu}. \\ \\
\begin{minipage}{\linewidth}
	\vspace{5mm}
	\begin{tikzpicture}[x=1.8cm, y=1.5cm]
	% Input layer
	\node[text width=3cm] at (-1,0) {Input ($23*40$)};
	\foreach \m in {2,3,...,22}
	\node [tdnn neuron] (input-\m) at (\m*0.25,0) {};
	
	\node [tdnn neuron, label=below:$x_{t - 13}$] (input-1) at (1*0.25,0) {};
	\node [tdnn neuron, label=below:$x_{t + 9}$] (input-23) at (23*0.25,0) {};
	
	% TDNN 1
	\node[text width=3cm] at (-1,0.5) {TDNN/LP 1};
	\node[text width=3cm] at (-1,1) {Hidden ($7*300$)};
	\foreach \m [count=\y] in {1,2,...,7}
	\node [tdnn neuron] (hidden-1-\m) at (\y*0.25*3,1) {};
	
	% Splice
	\node[text width=3cm] at (-1,1.5) {Splice};
	\node[text width=3cm] at (-1,2) {Hidden ($8*300$)};
	\foreach \m [count=\y] in {1,2,...,8}
	\node [tdnn neuron] (hidden-2-\m) at (\y*0.25*3 - 0.40,2) {};
	
	% TDNN 2
	\node[text width=3cm] at (-1,2.5) {TDNN/LP 2};
	\node[text width=3cm] at (-1,3) {Hidden ($4*300$)};
	\foreach \m [count=\y] in {1,2,...,4}
	\node [tdnn neuron] (hidden-3-\m) at (\y*0.25*6 - 0.80,3) {};
	
	% TDNN 3
	\node[text width=3cm] at (-1,3.5) {TDNN/LP 2};
	\node[text width=3cm] at (-1,4) {Hidden ($2*300$)};
	\foreach \m [count=\y] in {1,2}
	\node [tdnn neuron] (hidden-4-\m) at (\y*0.25*12 - 1.60,4) {};
	
	% TDNN 4
	\node[text width=3cm] at (-1,4.5) {TDNN/LP 2};
	\node[text width=3cm] at (-1,5) {Hidden ($1*400$)};
	\node [tdnn neuron] (hidden-5) at (1*0.25 + 11 * 0.25,5) {};
	
	% Classifier
	\node[text width=3cm] at (-1,5.5) {Linear/Softmax};
	\node[text width=3cm] at (-1,6) {Output ($1*8156$)};
	\node [tdnn neuron, label=above:$y_{t}$] (classify-1) at (1*0.25 + 11 * 0.25,6) {};
	
	% Edges
	%L1 Edges
	\foreach \m [
	evaluate=\m as \nstart using int(((\m - 1) * 3) + 1),
	evaluate=\m as \nstep using int(((\m - 1) * 3) + 2),
	evaluate=\m as \nend using int(((\m - 1)* 3) + 5)] in {1,2,...,7}
	\foreach \i in {\nstart,\nstep,...,\nend}
	\draw (input-\i.north) -- (hidden-1-\m);  
	
	%Splice Edge
	\draw (hidden-1-1.north) -- (hidden-2-1.south);  
	\draw (hidden-1-2.north) -- (hidden-2-2.south);
	\draw (hidden-1-3.north) -- (hidden-2-3.south);
	\draw (hidden-1-4.north) -- (hidden-2-4.south);
	\draw (hidden-1-4.north) -- (hidden-2-5.south);
	\draw (hidden-1-5.north) -- (hidden-2-6.south);
	\draw (hidden-1-6.north) -- (hidden-2-7.south);
	\draw (hidden-1-7.north) -- (hidden-2-8.south);
	
	% Edges
	%L2 Edges
	\foreach \m [
	evaluate=\m as \na using int(((\m - 1) * 2) + 1),
	evaluate=\m as \nb using int(((\m - 1) * 2) + 2)] in {1,2,...,4} {
		\draw (hidden-2-\na.north) -- (hidden-3-\m);
		\draw (hidden-2-\nb.north) -- (hidden-3-\m);
	}
	
	%Edges
	%L3 Edges
	\draw (hidden-3-1.north) -- (hidden-4-1);
	\draw (hidden-3-2.north) -- (hidden-4-1);
	\draw (hidden-3-3.north) -- (hidden-4-2);
	\draw (hidden-3-4.north) -- (hidden-4-2);
	
	%L4 edges
	\draw (hidden-4-1) -- (hidden-5);
	\draw (hidden-4-2) -- (hidden-5);

	% Classify Edges
	\draw (hidden-5) -- (classify-1);
	
	\end{tikzpicture}
	\captionof{figure}{Illustration of the final TDNN model in the time domain}
	\label{fig:final_tdnn}
	\vspace{5mm}
\end{minipage}
\subsection{Fully Connected Baseline Model}
We compare the results of our TDNN with a fully connected network that achieved comparable performance on an reverberated training set. The model consists of six linear layers with a width of 1600, followed by ReLU nonlinearities. The output layer is a linear layer followed by a Softmax nonlinearity. We tested input contexts of $(-13, 9)$ as well as $(-5, 5)$.
\section{Training Setup}
We used the same training set as described in chapter \ref{ch:tdnn_design} as well as a reverberated version of the training set. As in chapter \ref{ch:tdnn_design}, we used a SGD variant with newbob learning rate scheduling and momentum. The loss function was frame based cross entropy loss. The input features at each time frame were 40 log-mel coefficients as in \cite{nguyen20162016}. Our input features were mean normalized over the whole utterance with a mean of 0 and a variance of 2. This is different form the unnormalized 140 dimensional input vector used by \cite{peddinti2015reverberation} in \cite{peddinti2015jhu} for their TDNN model.
\section{Data Augmentation}
For training and evaluating acoustic models that are robust on reverberated data, we trained them on data that was reverberated as well. For this purpose, we used a collection of recorded room impulse responses to augment our data set. This follows the theoretical insight given in section \ref{sec:reverberation}: For each audio sequence in our data set, we pick a random room impulse response and convolute the two signals to from an augmented signal. \\ \\
The room impulse responses we used are similar as in \cite{ritter2016training}. However we did only use the RWCP \cite{nakamura2000acoustical}, OMNI \cite{stewart2010database} and ACE \cite{eaton2015ace} datasets. The AIR dataset was not used \cite{jeub2009binaural}, since the amplitude in the different recordings in the dataset change significantly.
Evaluating the effects of different signal amplitudes was not a goal of this work, as we can assume that the audio frontend used will always provide a reasonable gain. We therefore normalized the room impulse responses very carefully before performing the convolution based on their signal energy. For the ACE dataset, we found that the energy direct transmission path was very low compared to the noise. In this case, we amplified the direct transmission path to generate a meaningful result. \\ \\
Illustration \ref{fig:air_spectrogram} shows an audio sample that was reverberated with this method. 
\section{Evaluation Results}
We test the four-layer TDNN model with the input context $(-13, 9)$, as well as the fully-connected model with the input contexts $(-13, 9)$ and $(-5, 5)$. To do so, we train each model on the clean training set as described in chapter \ref{ch:tdnn_design}. We separately tune each model on a combination on the clean and reverberated training set, with a total length of 902~hours. Then, we tune the $l_p$, $l_z$, master beam and softmax temperature parameters using the development set mentioned in chapter \ref{ch:tdnn_design}. The development set was not augmented. \\ 
For this evaluation we use the \textit{tst2014} dataset, which was the evaluation dataset for the IWSLT 2014 conference \cite{cettolo2014report}. Similar to our development dataset, the evaluation dataset also consists of TED talks with a total length of 2.1~hours. 

TODO: Compare Each WERs.  \\

Reverb FF 13/9: 1050 \\
Reverb FF 5/5: 1051 \\
Reverbed TDNN: I83 \\

Clean TDNN: I12 \\
Clean FF 13/9: I90 \\
Clean FF 5/5: I92 \\

\iffalse
TODO: Describe how we generated reverbed data
TODO: Describe how we tested on reverbed data
TODO: Describe final architecture and results

This chapter should summarize and interpret the results. It should give a clear insight
about which methods did decrease the FER and WER on reverbed and unreverbed data, respectivley.
\fi