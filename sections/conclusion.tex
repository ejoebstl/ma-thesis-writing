%% LaTeX2e class for student theses
%% sections/conclusion.tex
%% 
%% Karlsruhe Institute of Technology
%% Institute for Program Structures and Data Organization
%% Chair for Software Design and Quality (SDQ)
%%
%% Dr.-Ing. Erik Burger
%% burger@kit.edu
%%
%% Version 1.3.2, 2017-08-01

\chapter{Conclusion}
\label{ch:Conclusion}
During our evaluation in chapter \ref{ch:results} we found that our TDNN model did not yield a significant improvement over a fully connected model when trained on reverberated data. We also found that a fully connected model with the same input context was capable of slightly outperforming our TDNN model on the reverberated data validation set. This results are difficult to generalize to TDNNs as a whole for the following reasons: 
\begin{itemize}
	\item In chapter \ref{ch:tdnn_design}, we tuned our TDNN on clean data, with the assumption that a TDNN model that performs well on clean data also performs well on reverberated data. 
	\item The room impulse response normalization given in \ref{ch:results} might have been too aggressive, thus negating the advantage of TDNNs. In general, it is hard to measure the comprehensibleness of reverberated audio objectively.  
	\item In literature \cite{peddinti2015jhu} \cite{peddinti2015reverberation}, sMBR training criteria are used for TDNN acoustic model training. It might be worth to investigate this training procedure more closely.  
\end{itemize}
While the question whether TDNNs or fully connected networks are better for this specific problem is still unanswered, we provide several interesting insights that can improve speech recognition systems:
\begin{itemize}
	\item The modified L2 pooling nonlinearity introduced in section \ref{sec:tdnn_nonlin} performed better than L2 pooling combined with normalization.
	\item Augmented data can be used to boost the performance of acoustic models, even when only clean audio is of concern, as shown in chapter \ref{ch:results}.
	\item The calculation of priors from the network output, as shown in section \ref{sec:tdnn_prior}, given a randomly sampled subset of the training data, was shown to outperform priors which were calculated from the training data set. 
\end{itemize}
Overall, we were able to show that acoustic models can be made robust against reverberation, if they are trained using reverberated data as well.