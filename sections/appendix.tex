%% LaTeX2e class for student theses
%% sections/apendix.tex
%% 
%% Karlsruhe Institute of Technology
%% Institute for Program Structures and Data Organization
%% Chair for Software Design and Quality (SDQ)
%%
%% Dr.-Ing. Erik Burger
%% burger@kit.edu
%%
%% Version 1.3.2, 2017-08-01


\chapter{Appendix}
\label{ch:appendix}
\section{Optimal Decoder Parameters}
The following table contains a summary of all decoder parameters we found to be optimal for the experiments in our evaluation. This might be useful for reproducing our results or for further experiments. The table also shows the achieved word error rate on the clean and reverberated development data set. \\ \\ 
\begin{minipage}{\linewidth}
	\centering
	\begin{tabular}{@{\extracolsep{4pt}}lcccclll@{}}
		\toprule
		Model Name      & Training Data & $l_p$ & $l_z$ & $mb$ &  $1/\tau$ & clean WER & rvb WER \\\cmidrule[1pt]{1-1}\cmidrule[1pt]{2-8}
		TDNN $(-13, 9)$ & clean         & -15   & 95    & 6     & 0.9       & 14.9 & 27.6 \\
		FC $(-13, 9)$   & clean         & -10   & 90    & 6     & 0.8       & 15.1 & 28.4 \\
		FC $(-5, 5)$    & clean         & -10   & 90    & 6     & 0.8       & 15.3 & 35.5  \\\cmidrule[1pt]{1-1}\cmidrule[1pt]{2-8}
		TDNN $(-13, 9)$ & clean + rvb   & -5    & 95    & 6     & 0.85      & 14.8 & 20.4 \\
		FC $(-13, 9)$   & clean + rvb   & -10   & 90    & 6     & 0.78      & 14.7 & 19.8 \\
		FC $(-5, 5)$    & clean + rvb   & -10   & 95    & 6     & 1.0       & 14.6 & 20.3\\
		\bottomrule
	\end{tabular}
	\captionof{table}{Optimal decoder parameters and word error rate on clean and reverberated development data set}
	\label{tbl:decoder_params}
\end{minipage} \\ \\
It should be noted that the parameters shown here are implementation specific to the Janus recognition toolkit \cite{finke1997karlsruhe}.